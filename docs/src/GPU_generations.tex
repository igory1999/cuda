\subsection{Generations of GPU cards}
\begin{frame}[fragile]
  \frametitle{GPU hardware: Generations of GPU cards}
\begin{itemize}
\item For each compute capability NVIDIA releases many cards 
  for different usage mode: gaming, graphics, laptops, desktops, tablets, data center, etc
\item They differ by such hardware resources as memory, number of SMs, number of registers, etc.
\item We shall list here only several recent data center quality cards used for heavy computations. 

{\tiny
\begin{center}
\begin{tabular}{ |r|p{1.0cm}|p{0.9cm}|p{0.25cm}|r|p{0.7cm}|p{0.9cm}|p{0.8cm}|p{0.8cm}| } 
 \hline
 Model & {Micro Architecture} & {Compute Capability} & SMs  & cores &  Memory (GB) & {Connection to CPU} & {Bandwidth to CPU (GB/s)} & {Bandwidth to memory (GB/s)} \\ \hline
 K80   & Kepler               & 3.7                  & 2x13 & 2x2496& 2x12         & PCIe                & 12                        & 2x240                        \\ \hline
 P100  & Pascal               & 6.0                  & 56   & 3584  & 12, 16       & PCIe, NVlink        & 32, 160                   & 549, 732                     \\ \hline
 V100  & Volta                & 7.0                  & 80   & 5120  & 16, 32       & PCIe, NVlink        & 32, 300                   & 900                          \\ \hline
\end{tabular}
\end{center}
}

\item Modern consumer level GPU cards found in laptops and desktop typically can also be used for GPU computing.
\end{itemize}
\end{frame}