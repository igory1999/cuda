\begin{frame}[fragile]
  \frametitle{CUDA programming: Lab 6: coalescing}
\begin{itemize}
\item One of the most important performance consideration in programming for
  CUDA-capable GPU architectures, according to the documentation, is the \mydef{coalescing} of global memory accesses. 
\item Global memory loads and stores by threads of a warp are coalesced by the device into as few as
  one transaction when certain access requirements are met.
\item The access requirements for coalescing depend on the compute capability of the device
\item For devices of compute capability 2.x, for example, the concurrent accesses of the threads of a warp will coalesce into a number of
  transactions equal to the number of cache lines necessary to service all of the threads
  of the warp.
\item By default, all accesses are cached through L1, which as 128-byte lines. For
  scattered access patterns, to reduce overfetch, it can sometimes be useful to cache only in
  L2, which caches shorter 32-byte segments.
\end{itemize}
\end{frame}

\begin{frame}[fragile]
  \frametitle{CUDA programming: Lab 6: coalescing}
\begin{itemize}
\item For devices of compute capability 3.x, accesses to global memory are cached only in L2;
  L1 is reserved for local memory accesses.
\item In the lab, we are trying to copy one array into another using coalesced, misaligned and strided access:
{\tiny
{\color{mycolorcode}
\begin{verbatim}
__global__ void copy1(float *a, float *b, int n)
{
  int id = threadIdx.x + blockDim.x * blockIdx.x;
  if(id < n)
    a[id] = b[id];
}

__global__ void copy2(float *a, float *b, int n, int offset)
{
  int id = threadIdx.x + blockDim.x * blockIdx.x;
  if(id < n)
    a[id] = b[(id + offset) % n];
}

__global__ void copy3(float *a, float *b, int n, int stride)
{
  int id = threadIdx.x + blockDim.x * blockIdx.x;
  if(id < n)
    a[id] = b[(id * stride) % n];
}
\end{verbatim}
}
}
\end{itemize}
\end{frame}


\begin{frame}[fragile]
  \frametitle{CUDA programming: Lab 6: coalescing}
\begin{itemize}
\item Probably because K80 has compute capability 3.7, 
  I do not see that much difference in coalesced, misaligned or strided access to global memory in the lab:
  \begin{itemize}
  \item Running time for coalleced access is 130 ms
  \item Running time for misaligned and strided access is 155 ms
  \end{itemize}
\end{itemize}
\end{frame}
