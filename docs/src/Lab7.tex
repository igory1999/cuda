\begin{frame}[fragile]
  \frametitle{CUDA programming: Lab 7: atomics}
\begin{itemize}
\item Suppose we want each thread to increment a counter:
{\color{mycolorcode}
\begin{verbatim}
__device__ int counter = 0;
__global__ void increment()
{
  counter++;
}
\end{verbatim}
}
where \mycode{counter} above is a global device variable initialize to 0.
\item We naively expect that at the end the counter will be equal to the number of threads.
\item We block size is 1024 and grid size is 1024. So the total number of threads is 1048576.
\item However, when we run the program, each time we get a different number: 54, 56, 53, ...
\item The problem is \mydef{race condition}.
\end{itemize}
\end{frame}


\begin{frame}[fragile]
  \frametitle{CUDA programming: Lab 7: atomics}
\begin{itemize}
\item Incrementing a counter is not an \mydef{atomic operation} but consists of several atomic operations: 
  \begin{itemize}
  \item first the current value of the counter needs to be read from global memory into the register of the thread
  \item then the thread needs to increment it in register
  \item finally it writes it back into global memory
  \end{itemize}
\item For a sequential program this is not a problem.
\item However, for a multithreaded program it is since the order of those operations is not defined. For example, it can be as follows:
  \begin{itemize}
  \item Thread 1 reads counter = 3 into register;
  \item Thread 2 reads counter = 3 into register;
  \item Thread 1 increments the register to 4;
  \item Thread 1 writes 4 into the global variable counter which is now 4;
  \item Thread 2 increments its register to 4;
  \item Thread 2 writes its register to the global variable counter which is now 4.
  \item However, we obviously want it to be 5.
  \end{itemize}
\end{itemize}
\end{frame}
  

\begin{frame}[fragile]
  \frametitle{CUDA programming: Lab 7: atomics}
\begin{itemize}
\item To handle such situation, CUDA provides atomic operations. For example, we can rewrite the previous kernel:
{\color{mycolorcode}
\begin{verbatim}
__global__ void increment()
{
  atomicAdd(&counter, 1);
}
\end{verbatim}
}
\item Now the counter behaves as expected since once one thread starts changing the counter, others 
  would have to wait until the operation is finished.
\item There are the following atomic functions available: \mycode{atomicAdd}, \mycode{atomicSub}, \mycode{atomicExch}, \mycode{atomicMin}, \mycode{atomicMax}, 
  \mycode{atomicInc}, \mycode{atomicDec}, \mycode{atomicCAS}, \mycode{atomicAnd}, \mycode{atomicOr}, \mycode{atomicXor}.
\item Unfortunately most of the atomic operations are provided for integers only.
\item One can use \mycode{atomicCAS} to create atomic operations for other data types.
\end{itemize}
\end{frame}


\begin{frame}[fragile]
  \frametitle{CUDA programming: Lab 7: atomics}
\begin{itemize}
\item \mycode{atomicCAS} stands for \mydef{atomic Compare And Swap}
{\tiny
{\color{mycolorcode}
\begin{verbatim}
int atomicCAS(int* address, int compare, int val);

unsigned int atomicCAS(unsigned int* address, unsigned int compare, unsigned int val);

unsigned long long int atomicCAS(unsigned long long int* address, 
                                 unsigned long long int compare, 
                                 unsigned long long int val);
\end{verbatim}
}
}
\item It reads the 32-bit or 64-bit word old located at the address \mycode{address} in global or shared
  memory, computes {\color{mycolorcode}\verb|(old == compare ? val : old)|} , and stores the result back
  to memory at the same address. 
\item These three operations are performed in one atomic
  transaction. The function returns \mycode{old}.
\end{itemize}
\end{frame}


\begin{frame}[fragile]
  \frametitle{CUDA programming: Lab 7: atomics}
\begin{itemize}
\item Here is how \mycode{myAtomicAdd} for doubles can be implemented using \mycode{atomicCAS}
{\tiny
{\color{mycolorcode}
\begin{verbatim}
__device__ double myAtomicAdd(double * address, double val)
{
  unsigned long long int * address_as_ull = 
    (unsigned long long int*)address;
  unsigned long long int old = *address_as_ull, assumed;
  do {
    assumed = old;
    old = atomicCAS(address_as_ull, assumed,
                    __double_as_longlong(val + __longlong_as_double(assumed)));
  } while (assumed != old);

  return __longlong_as_double(old);
}
\end{verbatim}
}
}
\item Notice: atomics introduce some communication overhead since a variable is locked for other threads until the thread 
  that aquired the lock finishes with it.
\end{itemize}
\end{frame}
