\subsection{Compute Capability}
\begin{frame}[fragile]
  \frametitle{GPU hardware: Compute Capability}
\begin{itemize}
\item {\color{mycolordef}Compute Capability} defines a set of features that the program can use
\item GPU cards are advertised to support particular compute capability
\item K80 has compute capability 3.7, while V100 is compute capability 7.0 and the latest Turing architecture supports 7.5
\item When you compile binary that would run on GPU, you specify on the features from what compute capability your code relies
\item You can also specify for what particular hardware you build the code and save binaries for different GPU cards in the same executable
\item Executable also contains so called {\color{mycolordef}PTX} code which is basically an assembler code for GPU hardware that supports the specified compute capability.
\item PTX allows {\color{mycolordef}JIT - just in time compilation}: if you try to run your binary on the card for which the code was not compiled it gets 
  recompiled at run time provided that the card is backward compatible
  with the specified compute capability.
\end{itemize}
\end{frame}
