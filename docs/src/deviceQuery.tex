\subsection{Lab 1: Querying hardware}
\begin{frame}[fragile]
  \frametitle{GPU hardware: Lab 1: querying hardware}
\begin{itemize}
\item To find out how many GPU cards are there in a node, what kind, driver version, what processes are running on which GPU card, use {\color{mycolorcli}nvidia-smi}:
{\tiny
{\color{mycolorcli}
\begin{verbatim}
+-----------------------------------------------------------------------------+
| NVIDIA-SMI 410.72       Driver Version: 410.72       CUDA Version: 10.0     |
|-------------------------------+----------------------+----------------------+
| GPU  Name        Persistence-M| Bus-Id        Disp.A | Volatile Uncorr. ECC |
| Fan  Temp  Perf  Pwr:Usage/Cap|         Memory-Usage | GPU-Util  Compute M. |
|===============================+======================+======================|
|   0  Tesla K80           On   | 00000000:08:00.0 Off |                    0 |
| N/A   27C    P8    27W / 149W |      0MiB / 11441MiB |      0%      Default |
+-------------------------------+----------------------+----------------------+
|   1  Tesla K80           On   | 00000000:09:00.0 Off |                    0 |
| N/A   32C    P8    28W / 149W |      0MiB / 11441MiB |      0%      Default |
+-------------------------------+----------------------+----------------------+
|   2  Tesla K80           On   | 00000000:88:00.0 Off |                    0 |
| N/A   28C    P8    25W / 149W |      0MiB / 11441MiB |      0%      Default |
+-------------------------------+----------------------+----------------------+
|   3  Tesla K80           On   | 00000000:89:00.0 Off |                    0 |
| N/A   33C    P8    29W / 149W |      0MiB / 11441MiB |      0%      Default |
+-------------------------------+----------------------+----------------------+
                                                                               
+-----------------------------------------------------------------------------+
| Processes:                                                       GPU Memory |
|  GPU       PID   Type   Process name                             Usage      |
|=============================================================================|
|  No running processes found                                                 |
+-----------------------------------------------------------------------------+
\end{verbatim}
}
}

\end{itemize}
\end{frame}


\begin{frame}[fragile]
  \frametitle{GPU hardware: Lab 1: querying hardware}
\begin{itemize}
\item To submit a job in batch to gpu2 node that would run {\color{mycolorcli}nvidia-smi}:
{\color{mycolorcli}
\begin{verbatim}
cd cuda/labs/1
sbatch nvidia-smi.batch
\end{verbatim}
}
\item If we have enough GPUs, we can log into the node and use it interactively:
{\color{mycolorcli}
\begin{verbatim}
sinteractive -p gpu2 --gres=gpu:1 --reservation=CUDA
\end{verbatim}
}
\item Set up the environment {\color{mycolorcli}\verb|source cuda/labs/env.sh|} 
\item Above you ask for one gpu card in one gpu2 node. {\color{mycolorcli}\verb|--reservation|} only exists for the duration
  of this workshop. Otherwise, do not use it.

\item One can run {\color{mycolorcli}nvidia-smi} continously, every few seconds, as {\color{mycolorcli}top}, using {\color{mycolorcli}-l} flag. 
\item This can be very useful to diagnose problems: 
  \begin{itemize}
  \item is the program using too much GPU memory?
  \item is the load on GPU too low? 
  \item is there hardware problem with GPU?
  \end{itemize}
\end{itemize}
\end{frame}

\begin{frame}[fragile]
  \frametitle{GPU hardware: Lab 1: querying hardware}
\begin{itemize}
\item Note: one can restrict a program to see only some of the available GPUs by setting 
{\color{mycolorcli}\verb|CUDA_VISIBLE_DEVICES|} to the list of visible GPU cards, separated by comma
\item When you submit a job to SLURM scheduler, it sets this variable for you, so that each user on a node sees only as many GPU cards as was requested
\item Otherwise, the program sees all GPU cards and can grab them whether their are needed or not, possibly blocking other users
\end{itemize}
\end{frame}


\begin{frame}[fragile]
  \frametitle{GPU hardware: Lab 1: querying hardware}
\begin{itemize}
\item One can query the detailed parameters of the device with {\color{mycolorcli}deviceQuery} program that comes with CUDA.
\end{itemize}
{\tiny
{\color{mycolorcli}
\begin{verbatim}
/software/cuda-10.0-el7-x86_64/samples/1_Utilities/deviceQuery/deviceQuery Starting...

 CUDA Device Query (Runtime API) version (CUDART static linking)

Detected 1 CUDA Capable device(s)

Device 0: "Tesla K80"
  CUDA Driver Version / Runtime Version          10.0 / 10.0
  CUDA Capability Major/Minor version number:    3.7
  Total amount of global memory:                 11441 MBytes (11996954624 bytes)
  (13) Multiprocessors, (192) CUDA Cores/MP:     2496 CUDA Cores
  GPU Max Clock rate:                            824 MHz (0.82 GHz)
  Memory Clock rate:                             2505 Mhz
  Memory Bus Width:                              384-bit
  L2 Cache Size:                                 1572864 bytes
  Maximum Texture Dimension Size (x,y,z)         1D=(65536), 2D=(65536, 65536), 3D=(4096, 4096, 4096)
  Maximum Layered 1D Texture Size, (num) layers  1D=(16384), 2048 layers
  Maximum Layered 2D Texture Size, (num) layers  2D=(16384, 16384), 2048 layers
  Total amount of constant memory:               65536 bytes
  Total amount of shared memory per block:       49152 bytes
  Total number of registers available per block: 65536
  Warp size:                                     32
  Maximum number of threads per multiprocessor:  2048
  Maximum number of threads per block:           1024
...
\end{verbatim}
}
}

\end{frame}


\begin{frame}[fragile]
  \frametitle{GPU hardware: Lab 1: querying hardware}
{\tiny
{\color{mycolorcli}
\begin{verbatim}
...
  Max dimension size of a thread block (x,y,z): (1024, 1024, 64)
  Max dimension size of a grid size    (x,y,z): (2147483647, 65535, 65535)
  Maximum memory pitch:                          2147483647 bytes
  Texture alignment:                             512 bytes
  Concurrent copy and kernel execution:          Yes with 2 copy engine(s)
  Run time limit on kernels:                     No
  Integrated GPU sharing Host Memory:            No
  Support host page-locked memory mapping:       Yes
  Alignment requirement for Surfaces:            Yes
  Device has ECC support:                        Enabled
  Device supports Unified Addressing (UVA):      Yes
  Device supports Compute Preemption:            No
  Supports Cooperative Kernel Launch:            No
  Supports MultiDevice Co-op Kernel Launch:      No
  Device PCI Domain ID / Bus ID / location ID:   0 / 9 / 0
  Compute Mode:
     < Default (multiple host threads can use ::cudaSetDevice() with device simultaneously) >

deviceQuery, CUDA Driver = CUDART, CUDA Driver Version = 10.0, CUDA Runtime Version = 10.0, NumDevs = 1
Result = PASS

\end{verbatim}
}
}

\end{frame}
