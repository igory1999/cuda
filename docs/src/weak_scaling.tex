\subsection{weak}
\begin{frame}[fragile]
  \frametitle{Scaling: weak}
\begin{itemize}
\item Often we are interested in a different question: given more processors, how much more work 
  can we do during the fixed amount of time? 
This is called {\color{mycolordef}weak scaling} and is described by {\color{mycolordef}Gustafson's law}.
\begin{equation*}
morework = \frac{P*N + (1 - P)}{P + (1 - P)} = P*N + (1 - P)
\end{equation*}
\item 9 women producing 1 baby in 1 month - strong scaling
\item 9 women producing 9 babies instead of 1 baby in 9 months - weak scaling
\item Again, linear increase of the amount of work with the number of available processors is usually too optimistic: 
  there will likely be bottlenecks in communication bandwidth, synchronization, etc. that would limit strong scaling.
  For example, if there are 4 lanes in the road, only 4 cars can travel in parallel.
\end{itemize}
\end{frame}
